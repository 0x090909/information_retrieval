% samplepaper.tex: springer.com
% modificato da MM 06/05/2018
%
\documentclass[runningheads]{llncs}
\usepackage[italian]{babel}
\usepackage{graphicx}
\usepackage{subfigure}
\usepackage{listings}
\usepackage{threeparttable}
% Used for displaying a sample figure. If possible, figure files
% should be included in EPS format.  If you use the hyperref package,
% please uncomment the following line to display URLs in blue roman
% font according to Springer's eBook style:
% \renewcommand\UrlFont{\color{blue}\rmfamily}
\begin{document}
%
\title{Da mettere alla fine degli esperimenti}
%
% \titlerunning{Abbreviated paper title}
% If the paper title is too long for the running head, you can set
% an abbreviated paper title here
%
\author{%
  Alessandro Stefani\inst{1} \and
  Caterina Buranelli\inst{2} \and
  Cristi Gutu\inst{3}}
%
\authorrunning{Alessandro Stefani, Caterina Buranelli, Cristi Gutu}
% First names are abbreviated in the running head.  If there are more
% than two authors, 'et al.' is used.
%
\institute{Corso di laurea in Statistica per le tecnologie e le scienze,
  matricola 1148387 \email{alessandro.stefani.6@studenti.unipd.it} \and Corso di laurea in Statistica per le tecnologie e le scienze,
    matricola 1163443 \email{caterina.buranelli@studenti.unipd.it} \and Corso di laurea in Statistica per le tecnologie e le scienze,
      matricola 1147351 \email{gheorghecristi.gutu@studenti.unipd.it}
  }
%
\maketitle
% typeset the header of the contribution
%
\begin{abstract}
DA FINIRE QUANDO ABBIAMO MESSO A POSTO LA SEZIONE ESPERIMENTI.
Questo progetto tratta la realizzazione attraverso il pacchetto \emph{Whoosh}, di un motore di ricerca
volto al reperimento di documenti della collezione sperimentale \emph{OHSUMED} indicizzata opportunamente.
Il progetto è anche corredato di un webserver che permette all'utente di interrogare il motore di ricerca
in forma interattiva attraverso un browser a scelta.
 \keywords{{\it
      Information Retrieval  \and IR \and statistica \and reperimento \and indicizzazione \and Whoosh \and Python \and webserver \and web \and interrogazione web}}
\end{abstract}

\section{Introduzione}
\label{sec:introduzione}

Lo scopo finale di un Sistema di Information Retrieval (IRS) e' quello di reperire documenti rilevanti relativi a una certa esigenza informativa\footnote{insieme delle circostanze in cui una persona ha un problema da risolvere o un compito da svolgere e richiede informazioni importanti, utili o necessarie per la risoluzione del problema o lo svolgimento del compito}; dunque i documenti sono il primo ingresso del sistema, mentre il secondo e' costitutito dalle interrogazioni; i documenti devono essere indicizzati e nell'indice creato, si andra' a effettuare la ricerca per reperire documenti rilevanti\footnote{la rilevanza e' la proprieta' che rende l'informazione importante, utile o necessaria a soddisfare l'esigenza informativa dell'utente}. Questa seconda parte e' detta reperimento e non si occupa solo di ricercare tra i documenti, ma anche di riordinare secondo un certo ordine di rilevanza. Cio' che descrive i documenti e cio' che descrive le interrogazioni deve essere confrontabile, infatti nei programmi di indicizzaizone e di reperimento si usa uno schema, che deve essere uguale in entrambi i casi.
L'indicizzazione e' un trade-off tra il  miglioramento della rapprerentazione del contenuto informativo dei documenti (efficacia) e la gestione degli indici (efficienza). Benche' ci sia sempre tensione tra queste due caratteristiche, ad oggi la questione piu' studiata e' quella dell'efficienza.

Esistono diversi modi per risolvere questo problema, nel nostro caso abbiamo ricercato la configurazione migliore tra un sistema di
reperimento con o senza uso di stopword e il tuning dei parametri dello schema di pesatura\footnote{funzione che assegna per ogni documento diversi livelli d’importanza dei termini mediante dei pesi, che possono variare con l’interrogazione} BM25F \cite{WBC}.




L'obiettivo principale della relazione \`e da una parte la
documentazione del progetto di un servizio di {IR} e dall'altra una
misura del grado in cui si sia riusciti a mettere in pratica i
contenuti della disciplina illustrati durante le lezioni.

A tal scopo la relazione dovr\`a illustrare nelle sezioni successive:
\begin{itemize}
\item i metodi di indicizzazione,
\item i modelli di reperimento,
\item l'interfaccia basata su un \textit{browser} per il {WWW}
\item i risultati della \emph{valutazione} condotta con la collezione
  sperimentale OHSUMED.
\end{itemize}
Il lettore della relazione \`e lo studente medio di un corso di laurea
in statistica al quale la relazione deve dare tutti gli strumenti per
comprendere il contenuto.  Ci si metta nei suoi panni e si scriva
tutto ci\`o e solo ci\`e che serve.  Chiedersi qual \`e il messaggio
che lo studente deve ``portarsi a casa'', esplicitarlo in questo
paragrafo e concentrarsi su quello nel resto della relazione.

L'introduzione della relazione deve servire al lettore a capire se
vale la pena continuare a leggere il resto.  Si possono riassumere i
contenuti delle sezioni successive e metterne in evidenza i punti
principali.  La relazione consiste di tre paragrafi principali dopo
questa introduzione e prima della bibliografia, per la quale si
suggerisce Bib\TeX\ se si scrive con \LaTeX.

\section{Base di partenza}
\label{sec:base-di-partenza}

La base di partenza \`e formata dai metodi documentati nei libri di
testo.  Si eviti di trascrivere pari pari, si cerchi piuttosto di
rielaborare i contenuti in modo da renderli \emph{coerenti} col resto
della relazione; in particolare, si descrivano tutti e solo i metodi
usati negli esperimenti e si eviti di parlare di quei metodi che poi
non sono stati usati; ad esempio, se si conducono degli esperimenti
con BM25F, si deve descrivere questo schema di pesatura in questa
sezione.
%
%\section{Metodi proposti}
%\label{sec:metodi-utilizzati}
%
%Nel caso in cui si siano sviluppati:
%\begin{itemize}
%\item modelli di reperimento,
%\item metodi di indicizzazione,
%\item schemi di pesatura o
%\item altri metodi o tecniche
%\end{itemize}
%propri, non documentati in libri di testo o altra letteratura, si
%scriva in questa sezione una descrizione accurata e completa.  Si
%mettano in evidenza le caratteristiche distintive dei propri
%contributi.  Se non si \`e proposto nulla di nuovo, si scriva
%\emph{Nessuno}.  In una delle ultime lezioni si vedr\`a come
%implementare delle proprie funzioni di reperimento e schemi di
%pesatura.

\section{Esperimenti}
\label{sec:esperimenti}


Per gli esperimenti si \`e utilizzata la collezione sperimentale
chiamata OHSUMED\footnote{ \url{https://bit.ly/2wpOynZ}}, che contiene circa 54711 documenti.
Gli esperimenti effettivi sono iniziati solamente dopo aver stabilito quale
fosse la baseline dalla quale partire, in altre parole ci siamo chiesti quale
fosse la base di partenza dalla quale  migliorare il nostro sistema
di ricerca.  \par

Affinch\`e Whoosh indicizzi una collezione di documenti, \`e necessario
specificare uno schema, con diversi campi, che cambiano a seconda dei casi;
per ogni campo si indica  nome e tipo. \par

\begin{lstlisting}
schema = Schema(docid      	= ID(stored=True),
		title      	= TEXT(stored=True),
		identifier	= ID(stored=True),
		terms 		= NGRAM(stored=True),
		authors      	= NGRAM(stored=True),
		abstract 	= TEXT(stored=True),
		publication	= TEXT(stored=True),
		source 		= TEXT(stored=True))
\end{lstlisting}
\begin{tablenotes}
      \small
      \item \bf Figura: Schema necessario all'indicizzazione dei documenti.
    \end{tablenotes}


Definito lo schema abbiamo sfruttato la configurazione strettamente necessaria senza alcun tuning per avere
un sistema di reperimento "minimale".

La configurazione baseline ideale \`e stata scelta in base al Mean Average Precision (M.A.P.),\cite{WBC_map} variando il parametro che indica quale schema di
pesatura usare nel processo di reperimento, tra cui TF\_IDF e BM25F .

 Il processo e il codice di indicizzazione sono facilmente comprensibili visionando il file  \emph{indicizzazione\_batch\_baseline.py}. \par
\lstset{
  language=bash,
  basicstyle=\ttfamily
}

Per eseguire l'indicizzazione  baseline \`e sufficiente lanciare il seguente script python con il comando:
\begin{lstlisting}
  python indicizzazione_batch_baseline.py \
  cartella_indice file_documenti.xml
\end{lstlisting}

Per eseguire il reperimento che poi produce il file treceval baseline basato sullo schema TF\_IDF \`e sufficiente lanciare
lo script python con il comando:

\begin{lstlisting}
  python reperimento_batch_baseline.py cartella_indice \
  file_query.xml 1 > reperimento_baseline.treceval
\end{lstlisting}

Per eseguire il reperimento che poi produce il file treceval baseline basato sullo schema BM25F \`e sufficiente lanciare
lo script python con il comando:

\begin{lstlisting}
  python reperimento_batch_baseline.py cartella_indice \
  file_query.xml 2 > reperimento_baseline.treceval
\end{lstlisting}

\subsection{Risultati baseline con schema di pesatura TF IDF}
Descrizione: "1 Campo" significa che il reperimento e' stato eseguito soltanto
valutando il campo title,   "2 Campi" title e abstract,  "3 Campi" title, abstract e terms.
\begin{table}
\centering
\begin{tabular}{lll}
\textbf{ 1 Campo }                 & \textbf{ 2 Campi }                 & \textbf{ 3 Campi }                  \\
---------------------------------- & ---------------------------------- & ----------------------------------  \\
 num\textit{q all 63 }             &  num\textit{q all 63 }             &  num\textit{q all 63 }              \\
 num\textit{ret all 37454 }        &  num\textit{ret all 57356 }        &  num\textit{ret all 58456 }         \\
 num\textit{rel all 670 }          &  num\textit{rel all 670 }          &  num\textit{rel all 670 }           \\
 num\textit{rel}ret all 305        &  num\textit{rel}ret all 380        &  num\textit{rel}ret all 382         \\
map all 0.0833                     & map all 0.0591                     & map all 0.0829
\end{tabular}
\begin{tablenotes}
      \small
      \item \bf Figura: Risultati treceval, nessuna manipolazione del testo, numero risultati restituiti per ogni query = 1000, pesatura TF IDF.
    \end{tablenotes}
\end{table}

\subsection{Risultati baseline con schema di pesatura BM25F}
Descrizione: "1 Campo" significa che il reperimento e' stato eseguito soltanto
valutando il campo title,   "2 Campi" title e abstract,  "3 Campi" title, abstract e terms.
\begin{table}
\centering
\begin{tabular}{lll}
\textbf{ 1 Campo }           & \textbf{ 2 Campi }           & \textbf{ 3 Campi }            \\
---------------------------------- & ---------------------------------- & ----------------------------------  \\
 num\textit{q all 63 }       &  num\textit{q all 63 }       &  num\textit{q all 63 }        \\
 num\textit{ret all 37454 }  &  num\textit{ret all 57356 }  &  num\textit{ret all 58456 }   \\
 num\textit{rel all 670 }    &  num\textit{rel all 670 }    &  num\textit{rel all 670 }     \\
 num\textit{rel}ret all 307  &  num\textit{rel}ret all 387  &  num\textit{rel}ret all 383   \\
map all 0.1073               & map all \bf 0.1289               & map all 0.1227
\end{tabular}
\begin{tablenotes}
      \small
      \item \bf Figura: Risultati treceval, nessuna manipolazione del testo, numero risultati restituiti per
ogni query = 1000, pesatura BM25F.
    \end{tablenotes}
\end{table}

Alla luce dei risultati si e' scelto come valori per i parametri baseline: \textit{Documenti rilevanti reperiti}:  387;
\textit{M.A.P}:  0.1289.

\subsection{Primo tentativo: uso delle stopword}

Le stopword\cite{WBC_stopword} sono parole che non portano informazione
significativa al contenuto informativo come congiunzioni, articoli, avverbi..\par

Inizialmente si \`e pensato di utilizzare le stopword generali della lingua inglese, per eliminare le parole che non
portano informazione significativa e di conseguenza aumentare il parametro di interesse M.A.P. I risultati sono stati accettabili, ma siamo riusciti
a migliorarli togliendo anche le stopword cliniche\cite{stopword_cliniche}, cio\`e  strettamente inerenti al contesto medico. Abbiamo ottenuto i seguenti risultati:
\begin{table}
\centering
\begin{tabular}{lll}
\textbf{ 1 Campo }           & \textbf{ 2 Campi }           & \textbf{ 3 Campi }            \\
---------------------------------- & ---------------------------------- & ----------------------------------  \\
 num\textit{q all 63 }       &  num\textit{q all 63 }       &  num\textit{q all 63 }        \\
 num\textit{ret all 54873 }  &  num\textit{ret all 612669 }  &  num\textit{ret all 61908 }   \\
 num\textit{rel all 670 }    &  num\textit{rel all 670 }    &  num\textit{rel all 670 }     \\
 num\textit{rel}ret all 477  &  num\textit{rel}ret all 570  &  num\textit{rel}ret all 538   \\
map all 0.2045               & map all \bf 0.2752               & map all 0.1665
\end{tabular}
\begin{tablenotes}
      \small
      \item \bf Figura: Risultati treceval, rimozione delle stopword cliniche, numero risultati restituiti per
ogni query = 1000, pesatura BM25F.
    \end{tablenotes}
\end{table}



%
%Inizialmente si \`e pensato che utilizzare le stopword generali della lingua inglese fosse sufficiente per eliminare le
%parole che non portano informazione, e di conseguenza aumentare il parametro di interesse M.A.P; i risultati
%non sono stati molto soddisfacenti, abbiamo quindi optato per l'utilizzo di stopword cliniche, cio\`e stopword utilizzate
%soltato in ambito clinico/medico \cite{stopword_cliniche}.

\begin{thebibliography}{8}

\bibitem{WBC}
W. Bruce Croft and Donald Metzler and Trevor Strohman. Search Engines: Information Retrieval in Practice. Addison Wesley, (2009), pp. 250-252

\bibitem{WBC_stopword}
W. Bruce Croft and Donald Metzler and Trevor Strohman. Search Engines: Information Retrieval in Practice. Addison Wesley, (2009), pp. 90

\bibitem{WBC_map}
W. Bruce Croft and Donald Metzler and Trevor Strohman. Search Engines: Information Retrieval in Practice. Addison Wesley, (2009), pp. 313

\bibitem{stopword_cliniche}
Discovering Related Clinical Concepts Using Large Amounts of Clinical Notes. Ganesan, Kavita and Lloyd, Shane and Sarkar,
 Vikren, (2016), pp. 27-33

\end{thebibliography}

\begin{figure}[h!]
  \subfigure[Alessandro Stefani]{
    \includegraphics[width=30mm,height=30mm]{aleste}}
  \subfigure[Caterina Buranelli]{
    \includegraphics[width=30mm,height=30mm]{caterina}}
  \subfigure[Cristi Gutu]{
    \includegraphics[width=30mm,height=30mm]{cristi}}
\end{figure}

\end{document}
