% samplepaper.tex: springer.com
% modificato da MM 06/05/2018
%
\documentclass[runningheads]{llncs}
\usepackage[italian]{babel}
\usepackage{graphicx}
\usepackage{subfigure}
% Used for displaying a sample figure. If possible, figure files
% should be included in EPS format.  If you use the hyperref package,
% please uncomment the following line to display URLs in blue roman
% font according to Springer's eBook style:
% \renewcommand\UrlFont{\color{blue}\rmfamily}
\begin{document}
%
\title{Organizzazione e contenuti della relazione di mini-progetto per il gruppo ACC}
%
% \titlerunning{Abbreviated paper title}
% If the paper title is too long for the running head, you can set
% an abbreviated paper title here
%
\author{%
  Alessandro Stefani\inst{1} \and
  Caterina Buranelli\inst{2} \and
  Cristi Gutu\inst{3}}
%
\authorrunning{P. Autore et al.}
% First names are abbreviated in the running head.  If there are more
% than two authors, 'et al.' is used.
%
\institute{Corso di laurea in Statistica per le tecnologie e le scienze,
  matricola 1148387 \email{alessandro.stefani.6@studenti.unipd.it} \and Corso di laurea in Statistica per le tecnologie e le scienze,
    matricola 1234567 \email{caterina.buranelli@studenti.unipd.it} \and Corso di laurea in Statistica per le tecnologie e le scienze,
      matricola 1147351 \email{gheorghecristi.gutu@studenti.unipd.it}
  }
%
\maketitle
% typeset the header of the contribution
%
\begin{abstract}
Questo progetto tratta la realizzazione attraverso il pacchetto \emph{Whoosh}, di un motore di ricerca
volto al reperimento di documenti della collezione sperimentale \emph{OHSUMED} indicizzata opportunamente.
Il progetto è anche corredato di un webserver che permette all'utente di interrogare il motore di ricerca
in forma interattiva attraverso un browser a scelta.
 \keywords{{\it
      Information Retrieval  \and IR \and statistica \and reperimento \and indicizzazione \and Whoosh \and Python \and webserver \and web \and interrogazione web}}
\end{abstract}

\section{Introduzione}
\label{sec:introduzione}

Lo scopo è di costruire un motore di ricerca che sia in grado di reperire i documenti rilevanti relativi a una esigenza informativa\footnote{insieme delle circostanze in cui unapersona ha un problema da risolvere o un compito da svolgeree richiede informazioni importanti, utili o necessarie per larisoluzione del problema o lo svolgimento del compito}
espressa dall'utente sotto forma di query, dalla collezione sperimentale Ohsumed.


Esistono diversi modi per risolvere questo problema, nel nostro caso abbiamo ricercato la configurazione migliore tra un sistema di
reperimento con o senza uso di stopwords e il tuning dei parametri dello schema di pesatura\footnote{funzione che assegna per ogni documento diversi livelli d’importanza dei termini mediante dei pesi, che possono variare con l’interrogazione} BM25F \cite{WBC}.




L'obiettivo principale della relazione \`e da una parte la
documentazione del progetto di un servizio di {IR} e dall'altra una
misura del grado in cui si sia riusciti a mettere in pratica i
contenuti della disciplina illustrati durante le lezioni.

A tal scopo la relazione dovr\`a illustrare nelle sezioni successive:
\begin{itemize}
\item i metodi di indicizzazione,
\item i modelli di reperimento,
\item l'interfaccia basata su un \textit{browser} per il {WWW}
\item i risultati della \emph{valutazione} condotta con la collezione
  sperimentale OHSUMED.
\end{itemize}
Il lettore della relazione \`e lo studente medio di un corso di laurea
in statistica al quale la relazione deve dare tutti gli strumenti per
comprendere il contenuto.  Ci si metta nei suoi panni e si scriva
tutto ci\`o e solo ci\`e che serve.  Chiedersi qual \`e il messaggio
che lo studente deve ``portarsi a casa'', esplicitarlo in questo
paragrafo e concentrarsi su quello nel resto della relazione.

L'introduzione della relazione deve servire al lettore a capire se
vale la pena continuare a leggere il resto.  Si possono riassumere i
contenuti delle sezioni successive e metterne in evidenza i punti
principali.  La relazione consiste di tre paragrafi principali dopo
questa introduzione e prima della bibliografia, per la quale si
suggerisce Bib\TeX\ se si scrive con \LaTeX.

\section{Base di partenza}
\label{sec:base-di-partenza}

La base di partenza \`e formata dai metodi documentati nei libri di
testo.  Si eviti di trascrivere pari pari, si cerchi piuttosto di
rielaborare i contenuti in modo da renderli \emph{coerenti} col resto
della relazione; in particolare, si descrivano tutti e solo i metodi
usati negli esperimenti e si eviti di parlare di quei metodi che poi
non sono stati usati; ad esempio, se si conducono degli esperimenti
con BM25F, si deve descrivere questo schema di pesatura in questa
sezione.

\section{Metodi proposti}
\label{sec:metodi-utilizzati}

Nel caso in cui si siano sviluppati:
\begin{itemize}
\item modelli di reperimento,
\item metodi di indicizzazione,
\item schemi di pesatura o
\item altri metodi o tecniche
\end{itemize}
propri, non documentati in libri di testo o altra letteratura, si
scriva in questa sezione una descrizione accurata e completa.  Si
mettano in evidenza le caratteristiche distintive dei propri
contributi.  Se non si \`e proposto nulla di nuovo, si scriva
\emph{Nessuno}.  In una delle ultime lezioni si vedr\`a come
implementare delle proprie funzioni di reperimento e schemi di
pesatura.

\section{Esperimenti}
\label{sec:esperimenti}

Si descriva la collezione sperimentale OHSUMED in termini di
dimensione e tipo di dati.  Si descrivano i risultati dei confronti
tra metodi di base e/o quelli proposti.  Cruciale \`e la descrizione
accurata degli esperimenti: essa deve permetterne la replicazione.

Il \textit{software}, sia quello a titolo esemplificativo dei concetti
che quello che realizza i propri metodi, va caricato su moodle nella
cartella messa a disposizione pi\`u avanti.  Nella relazione va
scritta la descrizione dell'algoritmo in modo preciso e completo da
permetterne la riproduzione.

\begin{thebibliography}{8}
\bibitem{ref_article1}
Autore, F.: Article title. Journal \textbf{2}(5), 99--110 (2016)

\bibitem{ref_lncs1}
Autore, F., Autore, S.: Title of a proceedings paper. In: Editor,
F., Editor, S. (eds.) CONFERENCE 2016, LNCS, vol. 9999, pp. 1--13.
Springer, Heidelberg (2016). \doi{10.10007/1234567890}

\bibitem{WBC}
W. Bruce Croft and Donald Metzler and Trevor Strohman. Search Engines: Information Retrieval in Practice. Addison Wesley, (2009), pp. 250-252

\bibitem{ref_proc1}
Autore, A.-B.: Contribution title. In: 9th International Proceedings
on Proceedings, pp. 1--2. Publisher, Location (2010)

\bibitem{ref_url1}
LNCS Homepage, \url{http://www.springer.com/lncs}. Last accessed 4
Oct 2017
\end{thebibliography}

\begin{figure}[h!]
  \subfigure[Alessandro Stefani]{
    \includegraphics[width=30mm,height=30mm]{aleste}}
  \subfigure[Caterina Buranelli]{
    \includegraphics[width=30mm,height=30mm]{caterina}}
  \subfigure[Cristi Gutu]{
    \includegraphics[width=30mm,height=30mm]{cristi}}
\end{figure}

\end{document}
